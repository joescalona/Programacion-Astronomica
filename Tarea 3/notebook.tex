
% Default to the notebook output style

    


% Inherit from the specified cell style.




    
\documentclass[11pt]{article}

    
    
    \usepackage[T1]{fontenc}
    % Nicer default font (+ math font) than Computer Modern for most use cases
    \usepackage{mathpazo}

    % Basic figure setup, for now with no caption control since it's done
    % automatically by Pandoc (which extracts ![](path) syntax from Markdown).
    \usepackage{graphicx}
    % We will generate all images so they have a width \maxwidth. This means
    % that they will get their normal width if they fit onto the page, but
    % are scaled down if they would overflow the margins.
    \makeatletter
    \def\maxwidth{\ifdim\Gin@nat@width>\linewidth\linewidth
    \else\Gin@nat@width\fi}
    \makeatother
    \let\Oldincludegraphics\includegraphics
    % Set max figure width to be 80% of text width, for now hardcoded.
    \renewcommand{\includegraphics}[1]{\Oldincludegraphics[width=.8\maxwidth]{#1}}
    % Ensure that by default, figures have no caption (until we provide a
    % proper Figure object with a Caption API and a way to capture that
    % in the conversion process - todo).
    \usepackage{caption}
    \DeclareCaptionLabelFormat{nolabel}{}
    \captionsetup{labelformat=nolabel}

    \usepackage{adjustbox} % Used to constrain images to a maximum size 
    \usepackage{xcolor} % Allow colors to be defined
    \usepackage{enumerate} % Needed for markdown enumerations to work
    \usepackage{geometry} % Used to adjust the document margins
    \usepackage{amsmath} % Equations
    \usepackage{amssymb} % Equations
    \usepackage{textcomp} % defines textquotesingle
    % Hack from http://tex.stackexchange.com/a/47451/13684:
    \AtBeginDocument{%
        \def\PYZsq{\textquotesingle}% Upright quotes in Pygmentized code
    }
    \usepackage{upquote} % Upright quotes for verbatim code
    \usepackage{eurosym} % defines \euro
    \usepackage[mathletters]{ucs} % Extended unicode (utf-8) support
    \usepackage[utf8x]{inputenc} % Allow utf-8 characters in the tex document
    \usepackage{fancyvrb} % verbatim replacement that allows latex
    \usepackage{grffile} % extends the file name processing of package graphics 
                         % to support a larger range 
    % The hyperref package gives us a pdf with properly built
    % internal navigation ('pdf bookmarks' for the table of contents,
    % internal cross-reference links, web links for URLs, etc.)
    \usepackage{hyperref}
    \usepackage{longtable} % longtable support required by pandoc >1.10
    \usepackage{booktabs}  % table support for pandoc > 1.12.2
    \usepackage[inline]{enumitem} % IRkernel/repr support (it uses the enumerate* environment)
    \usepackage[normalem]{ulem} % ulem is needed to support strikethroughs (\sout)
                                % normalem makes italics be italics, not underlines
    

    
    
    % Colors for the hyperref package
    \definecolor{urlcolor}{rgb}{0,.145,.698}
    \definecolor{linkcolor}{rgb}{.71,0.21,0.01}
    \definecolor{citecolor}{rgb}{.12,.54,.11}

    % ANSI colors
    \definecolor{ansi-black}{HTML}{3E424D}
    \definecolor{ansi-black-intense}{HTML}{282C36}
    \definecolor{ansi-red}{HTML}{E75C58}
    \definecolor{ansi-red-intense}{HTML}{B22B31}
    \definecolor{ansi-green}{HTML}{00A250}
    \definecolor{ansi-green-intense}{HTML}{007427}
    \definecolor{ansi-yellow}{HTML}{DDB62B}
    \definecolor{ansi-yellow-intense}{HTML}{B27D12}
    \definecolor{ansi-blue}{HTML}{208FFB}
    \definecolor{ansi-blue-intense}{HTML}{0065CA}
    \definecolor{ansi-magenta}{HTML}{D160C4}
    \definecolor{ansi-magenta-intense}{HTML}{A03196}
    \definecolor{ansi-cyan}{HTML}{60C6C8}
    \definecolor{ansi-cyan-intense}{HTML}{258F8F}
    \definecolor{ansi-white}{HTML}{C5C1B4}
    \definecolor{ansi-white-intense}{HTML}{A1A6B2}

    % commands and environments needed by pandoc snippets
    % extracted from the output of `pandoc -s`
    \providecommand{\tightlist}{%
      \setlength{\itemsep}{0pt}\setlength{\parskip}{0pt}}
    \DefineVerbatimEnvironment{Highlighting}{Verbatim}{commandchars=\\\{\}}
    % Add ',fontsize=\small' for more characters per line
    \newenvironment{Shaded}{}{}
    \newcommand{\KeywordTok}[1]{\textcolor[rgb]{0.00,0.44,0.13}{\textbf{{#1}}}}
    \newcommand{\DataTypeTok}[1]{\textcolor[rgb]{0.56,0.13,0.00}{{#1}}}
    \newcommand{\DecValTok}[1]{\textcolor[rgb]{0.25,0.63,0.44}{{#1}}}
    \newcommand{\BaseNTok}[1]{\textcolor[rgb]{0.25,0.63,0.44}{{#1}}}
    \newcommand{\FloatTok}[1]{\textcolor[rgb]{0.25,0.63,0.44}{{#1}}}
    \newcommand{\CharTok}[1]{\textcolor[rgb]{0.25,0.44,0.63}{{#1}}}
    \newcommand{\StringTok}[1]{\textcolor[rgb]{0.25,0.44,0.63}{{#1}}}
    \newcommand{\CommentTok}[1]{\textcolor[rgb]{0.38,0.63,0.69}{\textit{{#1}}}}
    \newcommand{\OtherTok}[1]{\textcolor[rgb]{0.00,0.44,0.13}{{#1}}}
    \newcommand{\AlertTok}[1]{\textcolor[rgb]{1.00,0.00,0.00}{\textbf{{#1}}}}
    \newcommand{\FunctionTok}[1]{\textcolor[rgb]{0.02,0.16,0.49}{{#1}}}
    \newcommand{\RegionMarkerTok}[1]{{#1}}
    \newcommand{\ErrorTok}[1]{\textcolor[rgb]{1.00,0.00,0.00}{\textbf{{#1}}}}
    \newcommand{\NormalTok}[1]{{#1}}
    
    % Additional commands for more recent versions of Pandoc
    \newcommand{\ConstantTok}[1]{\textcolor[rgb]{0.53,0.00,0.00}{{#1}}}
    \newcommand{\SpecialCharTok}[1]{\textcolor[rgb]{0.25,0.44,0.63}{{#1}}}
    \newcommand{\VerbatimStringTok}[1]{\textcolor[rgb]{0.25,0.44,0.63}{{#1}}}
    \newcommand{\SpecialStringTok}[1]{\textcolor[rgb]{0.73,0.40,0.53}{{#1}}}
    \newcommand{\ImportTok}[1]{{#1}}
    \newcommand{\DocumentationTok}[1]{\textcolor[rgb]{0.73,0.13,0.13}{\textit{{#1}}}}
    \newcommand{\AnnotationTok}[1]{\textcolor[rgb]{0.38,0.63,0.69}{\textbf{\textit{{#1}}}}}
    \newcommand{\CommentVarTok}[1]{\textcolor[rgb]{0.38,0.63,0.69}{\textbf{\textit{{#1}}}}}
    \newcommand{\VariableTok}[1]{\textcolor[rgb]{0.10,0.09,0.49}{{#1}}}
    \newcommand{\ControlFlowTok}[1]{\textcolor[rgb]{0.00,0.44,0.13}{\textbf{{#1}}}}
    \newcommand{\OperatorTok}[1]{\textcolor[rgb]{0.40,0.40,0.40}{{#1}}}
    \newcommand{\BuiltInTok}[1]{{#1}}
    \newcommand{\ExtensionTok}[1]{{#1}}
    \newcommand{\PreprocessorTok}[1]{\textcolor[rgb]{0.74,0.48,0.00}{{#1}}}
    \newcommand{\AttributeTok}[1]{\textcolor[rgb]{0.49,0.56,0.16}{{#1}}}
    \newcommand{\InformationTok}[1]{\textcolor[rgb]{0.38,0.63,0.69}{\textbf{\textit{{#1}}}}}
    \newcommand{\WarningTok}[1]{\textcolor[rgb]{0.38,0.63,0.69}{\textbf{\textit{{#1}}}}}
    
    
    % Define a nice break command that doesn't care if a line doesn't already
    % exist.
    \def\br{\hspace*{\fill} \\* }
    % Math Jax compatability definitions
    \def\gt{>}
    \def\lt{<}
    % Document parameters
    \title{Escalona\_Joaquin\_3}
    
    
    

    % Pygments definitions
    
\makeatletter
\def\PY@reset{\let\PY@it=\relax \let\PY@bf=\relax%
    \let\PY@ul=\relax \let\PY@tc=\relax%
    \let\PY@bc=\relax \let\PY@ff=\relax}
\def\PY@tok#1{\csname PY@tok@#1\endcsname}
\def\PY@toks#1+{\ifx\relax#1\empty\else%
    \PY@tok{#1}\expandafter\PY@toks\fi}
\def\PY@do#1{\PY@bc{\PY@tc{\PY@ul{%
    \PY@it{\PY@bf{\PY@ff{#1}}}}}}}
\def\PY#1#2{\PY@reset\PY@toks#1+\relax+\PY@do{#2}}

\expandafter\def\csname PY@tok@gd\endcsname{\def\PY@tc##1{\textcolor[rgb]{0.63,0.00,0.00}{##1}}}
\expandafter\def\csname PY@tok@gu\endcsname{\let\PY@bf=\textbf\def\PY@tc##1{\textcolor[rgb]{0.50,0.00,0.50}{##1}}}
\expandafter\def\csname PY@tok@gt\endcsname{\def\PY@tc##1{\textcolor[rgb]{0.00,0.27,0.87}{##1}}}
\expandafter\def\csname PY@tok@gs\endcsname{\let\PY@bf=\textbf}
\expandafter\def\csname PY@tok@gr\endcsname{\def\PY@tc##1{\textcolor[rgb]{1.00,0.00,0.00}{##1}}}
\expandafter\def\csname PY@tok@cm\endcsname{\let\PY@it=\textit\def\PY@tc##1{\textcolor[rgb]{0.25,0.50,0.50}{##1}}}
\expandafter\def\csname PY@tok@vg\endcsname{\def\PY@tc##1{\textcolor[rgb]{0.10,0.09,0.49}{##1}}}
\expandafter\def\csname PY@tok@vi\endcsname{\def\PY@tc##1{\textcolor[rgb]{0.10,0.09,0.49}{##1}}}
\expandafter\def\csname PY@tok@vm\endcsname{\def\PY@tc##1{\textcolor[rgb]{0.10,0.09,0.49}{##1}}}
\expandafter\def\csname PY@tok@mh\endcsname{\def\PY@tc##1{\textcolor[rgb]{0.40,0.40,0.40}{##1}}}
\expandafter\def\csname PY@tok@cs\endcsname{\let\PY@it=\textit\def\PY@tc##1{\textcolor[rgb]{0.25,0.50,0.50}{##1}}}
\expandafter\def\csname PY@tok@ge\endcsname{\let\PY@it=\textit}
\expandafter\def\csname PY@tok@vc\endcsname{\def\PY@tc##1{\textcolor[rgb]{0.10,0.09,0.49}{##1}}}
\expandafter\def\csname PY@tok@il\endcsname{\def\PY@tc##1{\textcolor[rgb]{0.40,0.40,0.40}{##1}}}
\expandafter\def\csname PY@tok@go\endcsname{\def\PY@tc##1{\textcolor[rgb]{0.53,0.53,0.53}{##1}}}
\expandafter\def\csname PY@tok@cp\endcsname{\def\PY@tc##1{\textcolor[rgb]{0.74,0.48,0.00}{##1}}}
\expandafter\def\csname PY@tok@gi\endcsname{\def\PY@tc##1{\textcolor[rgb]{0.00,0.63,0.00}{##1}}}
\expandafter\def\csname PY@tok@gh\endcsname{\let\PY@bf=\textbf\def\PY@tc##1{\textcolor[rgb]{0.00,0.00,0.50}{##1}}}
\expandafter\def\csname PY@tok@ni\endcsname{\let\PY@bf=\textbf\def\PY@tc##1{\textcolor[rgb]{0.60,0.60,0.60}{##1}}}
\expandafter\def\csname PY@tok@nl\endcsname{\def\PY@tc##1{\textcolor[rgb]{0.63,0.63,0.00}{##1}}}
\expandafter\def\csname PY@tok@nn\endcsname{\let\PY@bf=\textbf\def\PY@tc##1{\textcolor[rgb]{0.00,0.00,1.00}{##1}}}
\expandafter\def\csname PY@tok@no\endcsname{\def\PY@tc##1{\textcolor[rgb]{0.53,0.00,0.00}{##1}}}
\expandafter\def\csname PY@tok@na\endcsname{\def\PY@tc##1{\textcolor[rgb]{0.49,0.56,0.16}{##1}}}
\expandafter\def\csname PY@tok@nb\endcsname{\def\PY@tc##1{\textcolor[rgb]{0.00,0.50,0.00}{##1}}}
\expandafter\def\csname PY@tok@nc\endcsname{\let\PY@bf=\textbf\def\PY@tc##1{\textcolor[rgb]{0.00,0.00,1.00}{##1}}}
\expandafter\def\csname PY@tok@nd\endcsname{\def\PY@tc##1{\textcolor[rgb]{0.67,0.13,1.00}{##1}}}
\expandafter\def\csname PY@tok@ne\endcsname{\let\PY@bf=\textbf\def\PY@tc##1{\textcolor[rgb]{0.82,0.25,0.23}{##1}}}
\expandafter\def\csname PY@tok@nf\endcsname{\def\PY@tc##1{\textcolor[rgb]{0.00,0.00,1.00}{##1}}}
\expandafter\def\csname PY@tok@si\endcsname{\let\PY@bf=\textbf\def\PY@tc##1{\textcolor[rgb]{0.73,0.40,0.53}{##1}}}
\expandafter\def\csname PY@tok@s2\endcsname{\def\PY@tc##1{\textcolor[rgb]{0.73,0.13,0.13}{##1}}}
\expandafter\def\csname PY@tok@nt\endcsname{\let\PY@bf=\textbf\def\PY@tc##1{\textcolor[rgb]{0.00,0.50,0.00}{##1}}}
\expandafter\def\csname PY@tok@nv\endcsname{\def\PY@tc##1{\textcolor[rgb]{0.10,0.09,0.49}{##1}}}
\expandafter\def\csname PY@tok@s1\endcsname{\def\PY@tc##1{\textcolor[rgb]{0.73,0.13,0.13}{##1}}}
\expandafter\def\csname PY@tok@dl\endcsname{\def\PY@tc##1{\textcolor[rgb]{0.73,0.13,0.13}{##1}}}
\expandafter\def\csname PY@tok@ch\endcsname{\let\PY@it=\textit\def\PY@tc##1{\textcolor[rgb]{0.25,0.50,0.50}{##1}}}
\expandafter\def\csname PY@tok@m\endcsname{\def\PY@tc##1{\textcolor[rgb]{0.40,0.40,0.40}{##1}}}
\expandafter\def\csname PY@tok@gp\endcsname{\let\PY@bf=\textbf\def\PY@tc##1{\textcolor[rgb]{0.00,0.00,0.50}{##1}}}
\expandafter\def\csname PY@tok@sh\endcsname{\def\PY@tc##1{\textcolor[rgb]{0.73,0.13,0.13}{##1}}}
\expandafter\def\csname PY@tok@ow\endcsname{\let\PY@bf=\textbf\def\PY@tc##1{\textcolor[rgb]{0.67,0.13,1.00}{##1}}}
\expandafter\def\csname PY@tok@sx\endcsname{\def\PY@tc##1{\textcolor[rgb]{0.00,0.50,0.00}{##1}}}
\expandafter\def\csname PY@tok@bp\endcsname{\def\PY@tc##1{\textcolor[rgb]{0.00,0.50,0.00}{##1}}}
\expandafter\def\csname PY@tok@c1\endcsname{\let\PY@it=\textit\def\PY@tc##1{\textcolor[rgb]{0.25,0.50,0.50}{##1}}}
\expandafter\def\csname PY@tok@fm\endcsname{\def\PY@tc##1{\textcolor[rgb]{0.00,0.00,1.00}{##1}}}
\expandafter\def\csname PY@tok@o\endcsname{\def\PY@tc##1{\textcolor[rgb]{0.40,0.40,0.40}{##1}}}
\expandafter\def\csname PY@tok@kc\endcsname{\let\PY@bf=\textbf\def\PY@tc##1{\textcolor[rgb]{0.00,0.50,0.00}{##1}}}
\expandafter\def\csname PY@tok@c\endcsname{\let\PY@it=\textit\def\PY@tc##1{\textcolor[rgb]{0.25,0.50,0.50}{##1}}}
\expandafter\def\csname PY@tok@mf\endcsname{\def\PY@tc##1{\textcolor[rgb]{0.40,0.40,0.40}{##1}}}
\expandafter\def\csname PY@tok@err\endcsname{\def\PY@bc##1{\setlength{\fboxsep}{0pt}\fcolorbox[rgb]{1.00,0.00,0.00}{1,1,1}{\strut ##1}}}
\expandafter\def\csname PY@tok@mb\endcsname{\def\PY@tc##1{\textcolor[rgb]{0.40,0.40,0.40}{##1}}}
\expandafter\def\csname PY@tok@ss\endcsname{\def\PY@tc##1{\textcolor[rgb]{0.10,0.09,0.49}{##1}}}
\expandafter\def\csname PY@tok@sr\endcsname{\def\PY@tc##1{\textcolor[rgb]{0.73,0.40,0.53}{##1}}}
\expandafter\def\csname PY@tok@mo\endcsname{\def\PY@tc##1{\textcolor[rgb]{0.40,0.40,0.40}{##1}}}
\expandafter\def\csname PY@tok@kd\endcsname{\let\PY@bf=\textbf\def\PY@tc##1{\textcolor[rgb]{0.00,0.50,0.00}{##1}}}
\expandafter\def\csname PY@tok@mi\endcsname{\def\PY@tc##1{\textcolor[rgb]{0.40,0.40,0.40}{##1}}}
\expandafter\def\csname PY@tok@kn\endcsname{\let\PY@bf=\textbf\def\PY@tc##1{\textcolor[rgb]{0.00,0.50,0.00}{##1}}}
\expandafter\def\csname PY@tok@cpf\endcsname{\let\PY@it=\textit\def\PY@tc##1{\textcolor[rgb]{0.25,0.50,0.50}{##1}}}
\expandafter\def\csname PY@tok@kr\endcsname{\let\PY@bf=\textbf\def\PY@tc##1{\textcolor[rgb]{0.00,0.50,0.00}{##1}}}
\expandafter\def\csname PY@tok@s\endcsname{\def\PY@tc##1{\textcolor[rgb]{0.73,0.13,0.13}{##1}}}
\expandafter\def\csname PY@tok@kp\endcsname{\def\PY@tc##1{\textcolor[rgb]{0.00,0.50,0.00}{##1}}}
\expandafter\def\csname PY@tok@w\endcsname{\def\PY@tc##1{\textcolor[rgb]{0.73,0.73,0.73}{##1}}}
\expandafter\def\csname PY@tok@kt\endcsname{\def\PY@tc##1{\textcolor[rgb]{0.69,0.00,0.25}{##1}}}
\expandafter\def\csname PY@tok@sc\endcsname{\def\PY@tc##1{\textcolor[rgb]{0.73,0.13,0.13}{##1}}}
\expandafter\def\csname PY@tok@sb\endcsname{\def\PY@tc##1{\textcolor[rgb]{0.73,0.13,0.13}{##1}}}
\expandafter\def\csname PY@tok@sa\endcsname{\def\PY@tc##1{\textcolor[rgb]{0.73,0.13,0.13}{##1}}}
\expandafter\def\csname PY@tok@k\endcsname{\let\PY@bf=\textbf\def\PY@tc##1{\textcolor[rgb]{0.00,0.50,0.00}{##1}}}
\expandafter\def\csname PY@tok@se\endcsname{\let\PY@bf=\textbf\def\PY@tc##1{\textcolor[rgb]{0.73,0.40,0.13}{##1}}}
\expandafter\def\csname PY@tok@sd\endcsname{\let\PY@it=\textit\def\PY@tc##1{\textcolor[rgb]{0.73,0.13,0.13}{##1}}}

\def\PYZbs{\char`\\}
\def\PYZus{\char`\_}
\def\PYZob{\char`\{}
\def\PYZcb{\char`\}}
\def\PYZca{\char`\^}
\def\PYZam{\char`\&}
\def\PYZlt{\char`\<}
\def\PYZgt{\char`\>}
\def\PYZsh{\char`\#}
\def\PYZpc{\char`\%}
\def\PYZdl{\char`\$}
\def\PYZhy{\char`\-}
\def\PYZsq{\char`\'}
\def\PYZdq{\char`\"}
\def\PYZti{\char`\~}
% for compatibility with earlier versions
\def\PYZat{@}
\def\PYZlb{[}
\def\PYZrb{]}
\makeatother


    % Exact colors from NB
    \definecolor{incolor}{rgb}{0.0, 0.0, 0.5}
    \definecolor{outcolor}{rgb}{0.545, 0.0, 0.0}



    
    % Prevent overflowing lines due to hard-to-break entities
    \sloppy 
    % Setup hyperref package
    \hypersetup{
      breaklinks=true,  % so long urls are correctly broken across lines
      colorlinks=true,
      urlcolor=urlcolor,
      linkcolor=linkcolor,
      citecolor=citecolor,
      }
    % Slightly bigger margins than the latex defaults
    
    \geometry{verbose,tmargin=1in,bmargin=1in,lmargin=1in,rmargin=1in}
    
    

    \begin{document}
    
    
    \maketitle
    
    

    
    \section{Ejercicio 1}\label{ejercicio-1}

Debe escribir un programa que le pida a un usuario adivinar un nombre,
pero sólo tienen 3 posibilidades hasta que el programa se cierra.

\subsection{Solucion}\label{solucion}

\href{https://es.stackoverflow.com/questions/14454/python-extraer-objeto-al-azar-de-una-lista}{Elegir
aleatoriamente} -\/-\textgreater{} De aquí saqué la forma para elegir
aleatoriamente un elemento de una lista.

El programa a continuación elegirá aleatoriamente un elemento de la
lista y el usuario debe adivinarlo. Si no lo logra, el programa
imprimirá el nombre que ha elegido.

    \begin{Verbatim}[commandchars=\\\{\}]
{\color{incolor}In [{\color{incolor} }]:} \PY{c+c1}{\PYZsh{}IMPORTAR RANDOM PARA ELEGIR UN ELEMENTO AL AZAR}
        \PY{k+kn}{import} \PY{n+nn}{random} 
        
        \PY{n}{nombres} \PY{o}{=} \PY{p}{[}\PY{l+s+s1}{\PYZsq{}}\PY{l+s+s1}{Amelia}\PY{l+s+s1}{\PYZsq{}}\PY{p}{,}\PY{l+s+s1}{\PYZsq{}}\PY{l+s+s1}{Luis}\PY{l+s+s1}{\PYZsq{}}\PY{p}{,}\PY{l+s+s1}{\PYZsq{}}\PY{l+s+s1}{Jaime}\PY{l+s+s1}{\PYZsq{}}\PY{p}{,}\PY{l+s+s1}{\PYZsq{}}\PY{l+s+s1}{Neil}\PY{l+s+s1}{\PYZsq{}}\PY{p}{,}\PY{l+s+s1}{\PYZsq{}}\PY{l+s+s1}{Constanza}\PY{l+s+s1}{\PYZsq{}}\PY{p}{,}\PY{l+s+s1}{\PYZsq{}}\PY{l+s+s1}{Joaquin}\PY{l+s+s1}{\PYZsq{}}\PY{p}{,}\PY{l+s+s1}{\PYZsq{}}\PY{l+s+s1}{Yuuki}\PY{l+s+s1}{\PYZsq{}}\PY{p}{]}
        \PY{n}{nom\PYZus{}elegido}\PY{o}{=}\PY{n}{random}\PY{o}{.}\PY{n}{choice}\PY{p}{(}\PY{n}{nombres}\PY{p}{)}
        \PY{c+c1}{\PYZsh{}IMPRIMIR CONDICIONES DEL JUEGO}
        \PY{k}{print}\PY{p}{(}\PY{l+s+s1}{\PYZsq{}}\PY{l+s+s1}{\PYZhy{}\PYZhy{}\PYZhy{}\PYZhy{}\PYZhy{}\PYZhy{} ADIVINA EL NOMBRE \PYZhy{}\PYZhy{}\PYZhy{}\PYZhy{}\PYZhy{}\PYZhy{}}\PY{l+s+s1}{\PYZsq{}}\PY{p}{)}
        \PY{k}{print}\PY{p}{(}\PY{l+s+s1}{\PYZsq{}}\PY{l+s+se}{\PYZbs{}n}\PY{l+s+s1}{ Tendras solo 3 oportunidades}\PY{l+s+s1}{\PYZsq{}}\PY{p}{)}
        \PY{k}{print}\PY{p}{(}\PY{l+s+s1}{\PYZsq{}}\PY{l+s+s1}{Los nombres a elegir son = }\PY{l+s+s1}{\PYZsq{}}\PY{p}{)}
        \PY{k}{for} \PY{n}{i} \PY{o+ow}{in} \PY{n}{nombres}\PY{p}{:}
            \PY{k}{print} \PY{l+s+s1}{\PYZsq{}}\PY{l+s+s1}{*}\PY{l+s+s1}{\PYZsq{}}\PY{p}{,}\PY{n}{i}
        
        
        \PY{c+c1}{\PYZsh{}PEDIR UN NOMBRE}
        \PY{n}{n}\PY{o}{=}\PY{n+nb}{raw\PYZus{}input}\PY{p}{(}\PY{l+s+s1}{\PYZsq{}}\PY{l+s+se}{\PYZbs{}n}\PY{l+s+s1}{ Comencemos, elige un nombre = }\PY{l+s+s1}{\PYZsq{}}\PY{p}{)}
        \PY{c+c1}{\PYZsh{}CONTADOR (POSIBILIDADES) = 1}
        \PY{n}{count} \PY{o}{=} \PY{l+m+mi}{1}
        \PY{c+c1}{\PYZsh{}CICLO CON TOPE = 3}
        \PY{k}{while} \PY{n+nb+bp}{True}\PY{p}{:}
        \PY{c+c1}{\PYZsh{}SI EL USUARIO ADIVINA}
            \PY{k}{if} \PY{n}{n} \PY{o}{==} \PY{n}{nom\PYZus{}elegido}\PY{p}{:}
                \PY{k}{print} \PY{p}{(}\PY{l+s+s1}{\PYZsq{}}\PY{l+s+s1}{Correcto! has adivinado :)}\PY{l+s+s1}{\PYZsq{}}\PY{p}{)}
                \PY{k}{break}
        \PY{c+c1}{\PYZsh{}SINO, SE DESCUENTA UNA POSIBILIDAD (AGREGANDOLE A COUNT)   }
            \PY{k}{else}\PY{p}{:} 
                \PY{n}{count}\PY{o}{+}\PY{o}{=}\PY{l+m+mi}{1}
        \PY{c+c1}{\PYZsh{}NONE NECESARIO PARA QUE EL USUARIO VUELVA A INTENTAR}
                \PY{n}{n}\PY{o}{=}\PY{n+nb+bp}{None}
                \PY{n}{n}\PY{o}{=}\PY{n+nb}{raw\PYZus{}input}\PY{p}{(}\PY{l+s+s1}{\PYZsq{}}\PY{l+s+se}{\PYZbs{}n}\PY{l+s+s1}{ Nope, intentalo nuevamente,elige un nombre = }\PY{l+s+s1}{\PYZsq{}}\PY{p}{)}
        \PY{c+c1}{\PYZsh{}SI LLEGA A 3 POSIBILIDADES}
                \PY{k}{if} \PY{n}{count} \PY{o}{==} \PY{l+m+mi}{3}\PY{p}{:}
                    \PY{k}{print} \PY{p}{(}\PY{l+s+s1}{\PYZsq{}}\PY{l+s+s1}{Has ocupado tus 3 opciones}\PY{l+s+s1}{\PYZsq{}}\PY{p}{)}
                    \PY{k}{print} \PY{l+s+s1}{\PYZsq{}}\PY{l+s+s1}{El nombre era = }\PY{l+s+s1}{\PYZsq{}}\PY{p}{,}\PY{n}{nom\PYZus{}elegido}
                    \PY{k}{break} 
\end{Verbatim}


    \section{Ejercicio 2}\label{ejercicio-2}

    Arreglen el código adjunto para que haga lo esperado: solo deje
``pasar'' a personal con uno de los tres nombres especificados. Deben
usar como mucho solo una instancia de ``==``.

    Codigo adjunto:

\begin{verbatim}
import sys
print("Hello. Please enter your name:")
name = sys.stdin.readline().strip()
if name == "Ana" or "Maria" or “Itziar":
    print("Access granted.")
else:
    print("Access denied.")
\end{verbatim}

    \subsection{Solución}\label{soluciuxf3n}

La forma en que se me ha ocurrido hacer el problema es incluir los
nombres del personal a una lista y hacer un ciclo IF como una puerta de
entrada: si el nombre ingresado se encuentra en la lista, se abre la
puerta, sino, no.

    \begin{Verbatim}[commandchars=\\\{\}]
{\color{incolor}In [{\color{incolor} }]:} \PY{k+kn}{import} \PY{n+nn}{sys}
        \PY{c+c1}{\PYZsh{}LISTA CON NOMBRES PERMITIDOS}
        \PY{n}{personal}\PY{o}{=} \PY{p}{[}\PY{l+s+s1}{\PYZsq{}}\PY{l+s+s1}{Ana}\PY{l+s+s1}{\PYZsq{}}\PY{p}{,}\PY{l+s+s1}{\PYZsq{}}\PY{l+s+s1}{Maria}\PY{l+s+s1}{\PYZsq{}}\PY{p}{,}\PY{l+s+s1}{\PYZsq{}}\PY{l+s+s1}{Itziar}\PY{l+s+s1}{\PYZsq{}}\PY{p}{]}
        
        \PY{k}{print}\PY{p}{(}\PY{l+s+s2}{\PYZdq{}}\PY{l+s+s2}{Hello. Please enter your name:}\PY{l+s+s2}{\PYZdq{}}\PY{p}{)}
        
        \PY{n}{name} \PY{o}{=} \PY{n}{sys}\PY{o}{.}\PY{n}{stdin}\PY{o}{.}\PY{n}{readline}\PY{p}{(}\PY{p}{)}\PY{o}{.}\PY{n}{strip}\PY{p}{(}\PY{p}{)}
        \PY{c+c1}{\PYZsh{}SI EL NOMBRE SE ENCUENTRA EN LA LISTA, PERMITIR}
        \PY{k}{if} \PY{n}{name} \PY{o+ow}{in} \PY{n}{personal}\PY{p}{:}
            \PY{k}{print}\PY{p}{(}\PY{l+s+s1}{\PYZsq{}}\PY{l+s+s1}{Access granted.}\PY{l+s+s1}{\PYZsq{}}\PY{p}{)}
        \PY{c+c1}{\PYZsh{}NO PERMITIR }
        \PY{k}{else}\PY{p}{:}
            \PY{k}{print}\PY{p}{(}\PY{l+s+s1}{\PYZsq{}}\PY{l+s+s1}{Acces denied.}\PY{l+s+s1}{\PYZsq{}}\PY{p}{)}
\end{Verbatim}


    ~

\section{Ejercicio 3}\label{ejercicio-3}

    Sea x un número entero.

\begin{enumerate}
\def\labelenumi{\Alph{enumi})}
\item
  Describa un algoritmo a base de iteraciones para comprobar si x es un
  número primo.
\item
  Escriba un programa en Python que solicite un input de un número
  entero, y que usa el algoritmo de arriba para comprobar si este número
  es un número primo.
\item
  Escriba un programa para comprobar si un número es el cuadrado de un
  número primo. Osea, que la raíz del número ingresado sea un número
  primo.
\end{enumerate}

Nota: Pruebe al inicio si el número es un cuadrado de un número entero.
Si eso es verdad, pruebe también si este número es primo.

    \subsection{Solución apartado A)}\label{soluciuxf3n-apartado-a}

La funcion sqrt, se importa para ser usada más adelante.

    \begin{Verbatim}[commandchars=\\\{\}]
{\color{incolor}In [{\color{incolor} }]:} \PY{k+kn}{from} \PY{n+nn}{math} \PY{k+kn}{import} \PY{n}{sqrt} 
        \PY{c+c1}{\PYZsh{}APARTADO A)}
        \PY{c+c1}{\PYZsh{}\PYZhy{}\PYZhy{}\PYZhy{}\PYZhy{}\PYZhy{}\PYZhy{}\PYZhy{}\PYZhy{}\PYZhy{}\PYZhy{}\PYZhy{}\PYZhy{}\PYZhy{}\PYZhy{}\PYZhy{}\PYZhy{}\PYZhy{}\PYZhy{}\PYZhy{}\PYZhy{}\PYZhy{}\PYZhy{}\PYZhy{}\PYZhy{}\PYZhy{}}
        \PY{c+c1}{\PYZsh{}DEFINIMOS FUNCION QUE ARROJA SI UN VALOR ES PRIMO O NO }
        \PY{c+c1}{\PYZsh{}INPUT = X}
        \PY{c+c1}{\PYZsh{}SI X ES PRIMO, RETORNA 1}
        \PY{c+c1}{\PYZsh{}SI NO, RETORNA 0}
        
        \PY{k}{def} \PY{n+nf}{es\PYZus{}primo}\PY{p}{(}\PY{n}{x}\PY{p}{)}\PY{p}{:}
        \PY{c+c1}{\PYZsh{}INICIO LOOP DESDE 2 HASTA X\PYZhy{}1 }
            \PY{k}{for} \PY{n}{i} \PY{o+ow}{in} \PY{n+nb}{range}\PY{p}{(}\PY{l+m+mi}{2}\PY{p}{,}\PY{n}{x}\PY{p}{)}\PY{p}{:}
        
        \PY{c+c1}{\PYZsh{}LA FUNCION \PYZhy{}IF ANY(LISTA)\PYZhy{} LA APRENDI EN SOLOLEARN (SECCION \PYZdq{}MAS TIPOS \PYZgt{}\PYZgt{} FUNCIONES }
        \PY{c+c1}{\PYZsh{}UTILES\PYZdq{})}
        \PY{c+c1}{\PYZsh{}PARA QUE UN NUMERO SEA PRIMO, ESTE DEBE SER DIVISIBLE SOLO POR 1 Y EL MISMO}
        \PY{c+c1}{\PYZsh{}POR LO TANTO, SI EXISTIERA ALGUN i (IF ANY) QUE EL CUOCIENTE ENTRE EL NUMERO E i }
        \PY{c+c1}{\PYZsh{}SEA IGUAL A 0, EL NUMERO NO ES PRIMO (Y LA FUNCION RETORNARA 0)}
                \PY{k}{if} \PY{n+nb}{any}\PY{p}{(}\PY{p}{[}\PY{n}{x}\PY{o}{\PYZpc{}}\PY{k}{i}==0]):
                    \PY{k}{return} \PY{l+m+mi}{0} 
            \PY{k}{return} \PY{l+m+mi}{1}
\end{Verbatim}


    \subsection{Solución apartado B)}\label{soluciuxf3n-apartado-b}

    Esto va junto con el programa anterior (apartado A)

    \begin{Verbatim}[commandchars=\\\{\}]
{\color{incolor}In [{\color{incolor} }]:} \PY{n}{x} \PY{o}{=} \PY{n+nb}{int}\PY{p}{(}\PY{n+nb}{input}\PY{p}{(}\PY{l+s+s1}{\PYZsq{}}\PY{l+s+s1}{Ingresa un numero y dire si es primo o no = }\PY{l+s+s1}{\PYZsq{}}\PY{p}{)}\PY{p}{)}
        \PY{k}{if} \PY{n}{x}\PY{o}{==}\PY{l+m+mi}{1}\PY{p}{:}
            \PY{k}{print}\PY{p}{(}\PY{l+s+s1}{\PYZsq{}}\PY{l+s+s1}{No es primo :( }\PY{l+s+s1}{\PYZsq{}}\PY{p}{)}
        
        \PY{k}{elif} \PY{n}{x}\PY{o}{\PYZgt{}}\PY{l+m+mi}{0}\PY{p}{:}
            \PY{k}{if} \PY{n}{es\PYZus{}primo}\PY{p}{(}\PY{n}{x}\PY{p}{)} \PY{o}{==} \PY{l+m+mi}{1}\PY{p}{:}
                \PY{k}{print}\PY{p}{(}\PY{l+s+s1}{\PYZsq{}}\PY{l+s+s1}{Es primo!}\PY{l+s+s1}{\PYZsq{}}\PY{p}{)}
            \PY{k}{else}\PY{p}{:}
                \PY{k}{print}\PY{p}{(}\PY{l+s+s1}{\PYZsq{}}\PY{l+s+s1}{No es primo :( }\PY{l+s+s1}{\PYZsq{}}\PY{p}{)}
        \PY{k}{else}\PY{p}{:} 
            \PY{k}{print}\PY{p}{(}\PY{l+s+s1}{\PYZsq{}}\PY{l+s+s1}{Debe ser un numero entero y positivo!}\PY{l+s+s1}{\PYZsq{}}\PY{p}{)}
\end{Verbatim}


    \subsection{Solución apartado C)}\label{soluciuxf3n-apartado-c}

Esta parte del código va junto con el apartado B (no necesario) y A
(necesario). Se importa la función sqrt del módulo math para poder
realizar la operación raíz :)

    \begin{Verbatim}[commandchars=\\\{\}]
{\color{incolor}In [{\color{incolor} }]:} \PY{n}{y} \PY{o}{=} \PY{n+nb}{int}\PY{p}{(}\PY{n+nb}{input}\PY{p}{(}\PY{l+s+s1}{\PYZsq{}}\PY{l+s+s1}{Ingresa un numero y dire si es el cuadrado de un primo = }\PY{l+s+s1}{\PYZsq{}}\PY{p}{)}\PY{p}{)}
        
        \PY{c+c1}{\PYZsh{}SI NUMERO INGRESADO ES MAYOR A 0, SE SACA LA RAIZ DEL NUMERO}
        \PY{k}{if} \PY{n}{y}\PY{o}{\PYZgt{}}\PY{l+m+mi}{0}\PY{p}{:}
            \PY{n}{raiz}\PY{o}{=}\PY{n+nb}{int}\PY{p}{(}\PY{n}{sqrt}\PY{p}{(}\PY{n}{y}\PY{p}{)}\PY{p}{)}
        \PY{c+c1}{\PYZsh{}SINO, PEDIR QUE SEA POSITIVO}
        \PY{k}{else}\PY{p}{:} 
            \PY{k}{print}\PY{p}{(}\PY{l+s+s1}{\PYZsq{}}\PY{l+s+s1}{Debe ser positivo}\PY{l+s+s1}{\PYZsq{}}\PY{p}{)}
        \PY{c+c1}{\PYZsh{}1 POR CONVENIO, NO ES CONSIDERADO PRIMO }
        \PY{c+c1}{\PYZsh{}LO LEI AQUI: https://es.wikipedia.org/wiki/N\PYZpc{}C3\PYZpc{}BAmero\PYZus{}primo}
        \PY{k}{if} \PY{n}{raiz}\PY{o}{==}\PY{l+m+mi}{1}\PY{p}{:}
            \PY{k}{print}\PY{p}{(}\PY{l+s+s1}{\PYZsq{}}\PY{l+s+s1}{No lo es :( }\PY{l+s+s1}{\PYZsq{}}\PY{p}{)}
        \PY{c+c1}{\PYZsh{}SI LA RAIZ ES DISTINTO DE 1}
        \PY{k}{else}\PY{p}{:}
        \PY{c+c1}{\PYZsh{}SI LA RAIZ ES PRIMO (VER APARTADO A)}
            \PY{k}{if} \PY{n}{es\PYZus{}primo}\PY{p}{(}\PY{n}{raiz}\PY{p}{)} \PY{o}{==} \PY{l+m+mi}{1}\PY{p}{:}
                \PY{k}{print} \PY{l+s+s1}{\PYZsq{}}\PY{l+s+s1}{Lo es! del primo }\PY{l+s+s1}{\PYZsq{}}\PY{p}{,}\PY{n}{raiz}
            \PY{k}{else}\PY{p}{:}
                \PY{k}{print}\PY{p}{(}\PY{l+s+s1}{\PYZsq{}}\PY{l+s+s1}{No lo es :( }\PY{l+s+s1}{\PYZsq{}}\PY{p}{)}
\end{Verbatim}


    \section{Ejercicio 4:}\label{ejercicio-4}

En el siguiente ejercicio, escribiremos programas que usan algoritmos
diferentes para calcular la tercera raíz con una precisión de ≈ 0.01.

\begin{verbatim}
a) Escriba un programa que use un algoritmo a base de exhaustive enumeration para la
determinar la tercera raíz. ¿Cuántas iteraciones necesita para determinarla con la precisión
deseada para los números 25, 500 y 10000?

b) Escriba un programa que calcula la tercera raíz con el algoritmo de bisección. ¿Cuántas
iteraciones necesita para determinarla con la precisión deseada para los números 25, 500 y
10000?

c) Escriba un programa que calcula la tercera raíz con el método de Newton. ¿Cuantas
iteraciones necesita para determinarla con la precisión deseada para los números 25, 500 y
10000?
\end{verbatim}

    \subsection{Solución apartado A)}\label{soluciuxf3n-apartado-a}

Link para ver lectures3-4.pdf -\/-\textgreater{}
\href{https://www.dropbox.com/s/sbgviuit4x4bzjp/lecture3-4.pdf?dl=0}{dropbox}

    \begin{Verbatim}[commandchars=\\\{\}]
{\color{incolor}In [{\color{incolor} }]:} \PY{c+c1}{\PYZsh{}EXHAUSTIVE ENUMERATION}
        \PY{c+c1}{\PYZsh{}SE DEFINE FUNCION RAIZ CUBICA}
        \PY{c+c1}{\PYZsh{}DESCARADAMENTE COPIADO DEL MATERIAL ENVIADO POR PROFESORA}
        \PY{c+c1}{\PYZsh{}PUEDES ENCONTRAR EL ORIGINAL EN LECTURE3\PYZhy{}4, PAG 68}
        \PY{c+c1}{\PYZsh{}LOS PASOS MATEMATICOS LOS EXPLICA LA PROFESORA MUCHO MEJOR QUE YO}
        \PY{c+c1}{\PYZsh{}SIN EMBARGO TRATARE DE DEJAR MIS COMENTARIOS CON LO QUE EL PROGRAMA HACE}
        \PY{k}{def} \PY{n+nf}{raiz\PYZus{}cubica}\PY{p}{(}\PY{n}{x}\PY{p}{)}\PY{p}{:}
        \PY{c+c1}{\PYZsh{}PRECISION}
            \PY{n}{epsilon} \PY{o}{=} \PY{l+m+mf}{0.01} 
            \PY{n}{step} \PY{o}{=} \PY{n}{epsilon}\PY{o}{*}\PY{o}{*}\PY{l+m+mi}{3}
            \PY{n}{ans} \PY{o}{=} \PY{l+m+mf}{0.0}
        \PY{c+c1}{\PYZsh{}CONTADORA DE ITERACIONES}
            \PY{n}{count}\PY{o}{=}\PY{l+m+mi}{0}
        
            \PY{k}{while} \PY{n+nb}{abs}\PY{p}{(}\PY{n}{ans}\PY{o}{*}\PY{o}{*}\PY{l+m+mi}{3} \PY{o}{\PYZhy{}} \PY{n}{x}\PY{p}{)} \PY{o}{\PYZgt{}}\PY{o}{=} \PY{n}{epsilon} \PY{o+ow}{and} \PY{n}{ans} \PY{o}{\PYZlt{}}\PY{o}{=}\PY{n}{x}\PY{p}{:}
        \PY{c+c1}{\PYZsh{}AGREGAR ITERACION HASTA QUE SE ROMPA EL CICLO}
                \PY{n}{count}\PY{o}{+}\PY{o}{=}\PY{l+m+mi}{1}
                \PY{n}{ans} \PY{o}{+}\PY{o}{=} \PY{n}{step}
            \PY{k}{if} \PY{n+nb}{abs}\PY{p}{(}\PY{n}{ans}\PY{o}{*}\PY{o}{*}\PY{l+m+mi}{3} \PY{o}{\PYZhy{}} \PY{n}{x}\PY{p}{)} \PY{o}{\PYZgt{}}\PY{o}{=} \PY{n}{epsilon}\PY{p}{:}
                \PY{k}{print} \PY{l+s+s1}{\PYZsq{}}\PY{l+s+s1}{No hemos encontrado la raiz de }\PY{l+s+s1}{\PYZsq{}}\PY{p}{,}\PY{n}{x}
        
            \PY{k}{print} \PY{l+s+s1}{\PYZsq{}}\PY{l+s+s1}{* La raiz de}\PY{l+s+s1}{\PYZsq{}}\PY{p}{,} \PY{n}{x}\PY{p}{,} \PY{l+s+s1}{\PYZsq{}}\PY{l+s+s1}{es aproximadamente}\PY{l+s+s1}{\PYZsq{}}\PY{p}{,} \PY{n}{ans}
            \PY{k}{print} \PY{l+s+s1}{\PYZsq{}}\PY{l+s+s1}{* Ocurrieron}\PY{l+s+s1}{\PYZsq{}}\PY{p}{,}\PY{n}{count}\PY{p}{,}\PY{l+s+s1}{\PYZsq{}}\PY{l+s+s1}{iteraciones }\PY{l+s+se}{\PYZbs{}n}\PY{l+s+s1}{\PYZsq{}}
        
        \PY{n}{raiz\PYZus{}cubica}\PY{p}{(}\PY{l+m+mi}{25}\PY{p}{)}
        \PY{n}{raiz\PYZus{}cubica}\PY{p}{(}\PY{l+m+mi}{500}\PY{p}{)}
        \PY{n}{raiz\PYZus{}cubica}\PY{p}{(}\PY{l+m+mi}{10000}\PY{p}{)}
\end{Verbatim}


    El programa arroja este resultado:

\begin{verbatim}
* La raiz de 25 es aproximadamente 2.92362800005. 
* Ocurrieron 2923628 iteraciones 

* La raiz de 500 es aproximadamente 7.93695300076
* Ocurrieron 7936953 iteraciones 

* La raiz de 10000 es aproximadamente 21.5443400005
* Ocurrieron 21544340 iteraciones 
\end{verbatim}

    \subsection{Solución apartado B)}\label{soluciuxf3n-apartado-b}

De
\href{https://www.pybonacci.org/2012/04/18/ecuaciones-no-lineales-metodo-de-biseccion-y-metodo-de-newton-en-python/}{aquí}
pude entender mas o menos el algoritmo planteado por la profesora. Me
parece bastante interesante este método de programación.

    \begin{Verbatim}[commandchars=\\\{\}]
{\color{incolor}In [{\color{incolor} }]:} \PY{c+c1}{\PYZsh{}BISECCION}
        \PY{c+c1}{\PYZsh{}DESCARADAMENTE COPIADO DEL MATERIAL ENVIADO POR PROFESORA}
        \PY{c+c1}{\PYZsh{}PUEDES ENCONTRAR EL ORIGINAL EN LECTURE3\PYZhy{}4, PAG 81}
        \PY{c+c1}{\PYZsh{}SE DEFINE FUNCION RAIZ CUBICA}
        \PY{k}{def} \PY{n+nf}{raiz\PYZus{}cubica}\PY{p}{(}\PY{n}{x}\PY{p}{)}\PY{p}{:}
        \PY{c+c1}{\PYZsh{}PRECISION}
            \PY{n}{epsilon} \PY{o}{=} \PY{l+m+mf}{0.01}
        \PY{c+c1}{\PYZsh{}MINIMO Y MAXIMO DEL INTERVALO}
            \PY{n}{low}\PY{p}{,}\PY{n}{high} \PY{o}{=} \PY{l+m+mf}{0.0}\PY{p}{,} \PY{n+nb}{max}\PY{p}{(}\PY{l+m+mf}{1.0}\PY{p}{,}\PY{n}{x}\PY{p}{)}
        \PY{c+c1}{\PYZsh{}RESPUESTA}
            \PY{n}{ans} \PY{o}{=} \PY{p}{(}\PY{n}{high} \PY{o}{+} \PY{n}{low}\PY{p}{)}\PY{o}{/}\PY{l+m+mf}{2.0} 
        \PY{c+c1}{\PYZsh{}CONTADORA DE ITERACIONES}
            \PY{n}{count} \PY{o}{=} \PY{l+m+mi}{0}
        
            \PY{k}{while} \PY{n+nb}{abs}\PY{p}{(}\PY{n}{ans}\PY{o}{*}\PY{o}{*}\PY{l+m+mi}{3} \PY{o}{\PYZhy{}} \PY{n}{x}\PY{p}{)} \PY{o}{\PYZgt{}}\PY{o}{=} \PY{n}{epsilon}\PY{p}{:}
                \PY{c+c1}{\PYZsh{}print \PYZsq{}low =\PYZsq{},low, \PYZsq{}high =\PYZsq{},high, \PYZsq{}ans = \PYZsq{},ans}
                \PY{n}{count} \PY{o}{+}\PY{o}{=}\PY{l+m+mi}{1}
        \PY{c+c1}{\PYZsh{}SI EL CUADRADO DE LA POSIBLE RESPUESTA ES MENOR QUE X }
        \PY{c+c1}{\PYZsh{}ENTONCES DEBE ESTAR A LA IZQUIERDA}
                \PY{k}{if} \PY{n}{ans}\PY{o}{*}\PY{o}{*}\PY{l+m+mi}{3} \PY{o}{\PYZlt{}} \PY{n}{x}\PY{p}{:}
                    \PY{n}{low} \PY{o}{=} \PY{n}{ans}
        \PY{c+c1}{\PYZsh{}SI ES MAYOR, DEBE ESTAR A LA DERECHA}
                \PY{k}{else}\PY{p}{:}
                    \PY{n}{high} \PY{o}{=} \PY{n}{ans}
        
                \PY{n}{ans} \PY{o}{=} \PY{p}{(}\PY{n}{high} \PY{o}{+} \PY{n}{low}\PY{p}{)}\PY{o}{/}\PY{l+m+mf}{2.0}
            \PY{k}{print} \PY{l+s+s1}{\PYZsq{}}\PY{l+s+s1}{* La raiz de}\PY{l+s+s1}{\PYZsq{}}\PY{p}{,} \PY{n}{x}\PY{p}{,} \PY{l+s+s1}{\PYZsq{}}\PY{l+s+s1}{es aproximadamente}\PY{l+s+s1}{\PYZsq{}}\PY{p}{,} \PY{n}{ans}
            \PY{k}{print} \PY{l+s+s1}{\PYZsq{}}\PY{l+s+s1}{* Ocurrieron}\PY{l+s+s1}{\PYZsq{}}\PY{p}{,}\PY{n}{count}\PY{p}{,}\PY{l+s+s1}{\PYZsq{}}\PY{l+s+s1}{iteraciones }\PY{l+s+se}{\PYZbs{}n}\PY{l+s+s1}{\PYZsq{}}
        \PY{n}{raiz\PYZus{}cubica}\PY{p}{(}\PY{l+m+mi}{25}\PY{p}{)}
        \PY{n}{raiz\PYZus{}cubica}\PY{p}{(}\PY{l+m+mi}{500}\PY{p}{)}
        \PY{n}{raiz\PYZus{}cubica}\PY{p}{(}\PY{l+m+mi}{10000}\PY{p}{)}
\end{Verbatim}


    El programa arroja este resultado:

\begin{verbatim}
* La raiz de 25 es aproximadamente 2.92434692383
* Ocurrieron 14 iteraciones

* La raiz de 500 es aproximadamente 7.93695449829
* Ocurrieron 19 iteraciones 

* La raiz de 10000 es aproximadamente 21.5443409979
* Ocurrieron 28 iteraciones 
\end{verbatim}

    \subsection{Solución apartado C)}\label{soluciuxf3n-apartado-c}

    \begin{Verbatim}[commandchars=\\\{\}]
{\color{incolor}In [{\color{incolor} }]:} \PY{c+c1}{\PYZsh{}NEWTON}
        \PY{c+c1}{\PYZsh{}DESCARADAMENTE COPIADO DEL MATERIAL ENVIADO POR PROFESORA}
        \PY{c+c1}{\PYZsh{}PUEDES ENCONTRAR EL ORIGINAL EN LECTURE3\PYZhy{}4, PAG 89}
        \PY{c+c1}{\PYZsh{}SE DEFINE FUNCION RAIZ CUBICA}
        \PY{k}{def} \PY{n+nf}{raiz\PYZus{}cubica}\PY{p}{(}\PY{n}{x}\PY{p}{)}\PY{p}{:}
        \PY{c+c1}{\PYZsh{}PRECISION}
            \PY{n}{epsilon} \PY{o}{=} \PY{l+m+mf}{0.01}
        \PY{c+c1}{\PYZsh{}RESPUEST}
            \PY{n}{guess} \PY{o}{=} \PY{n}{x}\PY{o}{/}\PY{l+m+mf}{2.0}
        \PY{c+c1}{\PYZsh{}CONTADORA DE ITERACIONES}
            \PY{n}{count}\PY{o}{=}\PY{l+m+mi}{0}
            \PY{k}{while} \PY{n+nb}{abs}\PY{p}{(}\PY{n}{guess}\PY{o}{*}\PY{o}{*}\PY{l+m+mi}{3} \PY{o}{\PYZhy{}} \PY{n}{x}\PY{p}{)} \PY{o}{\PYZgt{}}\PY{o}{=} \PY{n}{epsilon}\PY{p}{:}
                \PY{n}{count}\PY{o}{+}\PY{o}{=} \PY{l+m+mi}{1}
        \PY{c+c1}{\PYZsh{}METODO DE NEWTON }
                \PY{n}{guess} \PY{o}{=} \PY{n}{guess} \PY{o}{\PYZhy{}} \PY{p}{(}\PY{p}{(}\PY{p}{(}\PY{n}{guess}\PY{o}{*}\PY{o}{*}\PY{l+m+mi}{3}\PY{p}{)} \PY{o}{\PYZhy{}} \PY{n}{x}\PY{p}{)} \PY{o}{/} \PY{p}{(}\PY{l+m+mi}{3}\PY{o}{*}\PY{p}{(}\PY{n}{guess}\PY{o}{*}\PY{o}{*}\PY{l+m+mi}{2}\PY{p}{)}\PY{p}{)}\PY{p}{)}
            \PY{c+c1}{\PYZsh{}print(\PYZsq{}* Ocurrieron \PYZsq{}+str(count)+ \PYZsq{} iteraciones \PYZbs{}n\PYZsq{})}
            \PY{k}{print} \PY{l+s+s1}{\PYZsq{}}\PY{l+s+s1}{* La raiz de}\PY{l+s+s1}{\PYZsq{}}\PY{p}{,} \PY{n}{x}\PY{p}{,} \PY{l+s+s1}{\PYZsq{}}\PY{l+s+s1}{es aproximadamente}\PY{l+s+s1}{\PYZsq{}}\PY{p}{,} \PY{n}{guess}
            \PY{k}{print} \PY{l+s+s1}{\PYZsq{}}\PY{l+s+s1}{* Ocurrieron}\PY{l+s+s1}{\PYZsq{}}\PY{p}{,}\PY{n}{count}\PY{p}{,}\PY{l+s+s1}{\PYZsq{}}\PY{l+s+s1}{iteraciones }\PY{l+s+se}{\PYZbs{}n}\PY{l+s+s1}{\PYZsq{}}
        \PY{n}{raiz\PYZus{}cubica}\PY{p}{(}\PY{l+m+mi}{25}\PY{p}{)}
        \PY{n}{raiz\PYZus{}cubica}\PY{p}{(}\PY{l+m+mi}{500}\PY{p}{)}
        \PY{n}{raiz\PYZus{}cubica}\PY{p}{(}\PY{l+m+mi}{10000}\PY{p}{)}
\end{Verbatim}


    El programa arroja este resultado:

\begin{verbatim}
* La raiz de 25 es aproximadamente 2.9242328368
* Ocurrieron 6 iteraciones 

* La raiz de 500 es aproximadamente 7.93700527704
* Ocurrieron 12 iteraciones 

* La raiz de 10000 es aproximadamente 21.5443469166
* Ocurrieron 17 iteraciones 
\end{verbatim}

    \section{Ejercicio 5}\label{ejercicio-5}

    Ejercicio 5: Usted tiene las siguientes ecuaciones:

\begin{enumerate}
\def\labelenumi{\alph{enumi})}
\item
  \(x^2 = 4x\)
\item
  \(e^x = 4x\)
\item
  \(10x = x^2\)
\end{enumerate}

Para cada ecuación, defina una función \(f(x)\) de forma que el cero de
la función f sea la solución de la ecuación. Después, calcule también
\(df/dx\) y use el método de Newton para determinar la solución.
\textbf{(No nos interesa la solución trivial x = 0 en caso de a) y c).
Si resulta zero, cambie el supuesto inicial.)}

    \subsection{Solución apartado A)}\label{soluciuxf3n-apartado-a}

    Ésta solución y las que siguen, han sido basadas tras leer la entrada de
\emph{Shah} en
\href{https://pythonfornumericalmethods.wordpress.com/2015/10/17/metodo-de-newton/}{este
link}

    \begin{Verbatim}[commandchars=\\\{\}]
{\color{incolor}In [{\color{incolor} }]:} \PY{c+c1}{\PYZsh{}definimos funciones}
        
        \PY{c+c1}{\PYZsh{} f1(x)\PYZhy{}\PYZhy{}\PYZhy{}\PYZgt{} x\PYZca{}2 \PYZhy{} 4x }
        \PY{c+c1}{\PYZsh{} f1\PYZsq{}(x)\PYZhy{}\PYZhy{}\PYZhy{}\PYZgt{} 2x \PYZhy{} 4}
        \PY{k}{def} \PY{n+nf}{f1}\PY{p}{(}\PY{n}{x}\PY{p}{)}\PY{p}{:}
            \PY{k}{return} \PY{n}{x}\PY{o}{*}\PY{o}{*}\PY{l+m+mi}{2} \PY{o}{\PYZhy{}} \PY{l+m+mi}{4}\PY{o}{*}\PY{n}{x}
        
        \PY{k}{def} \PY{n+nf}{df1}\PY{p}{(}\PY{n}{x}\PY{p}{)}\PY{p}{:}
            \PY{k}{return} \PY{l+m+mi}{2}\PY{o}{*}\PY{n}{x}\PY{o}{\PYZhy{}}\PY{l+m+mi}{4}
        
        \PY{c+c1}{\PYZsh{}supuesto inicial (con 2.0 retorna ZeroDivisionError)}
        \PY{n}{xnew} \PY{o}{=} \PY{p}{[}\PY{l+m+mf}{3.0}\PY{p}{]}
        \PY{c+c1}{\PYZsh{}error tolerado al aproximar }
        \PY{n}{erro} \PY{o}{=} \PY{l+m+mf}{0.001} 
        \PY{c+c1}{\PYZsh{}raiz de la funcion}
        \PY{n}{resp} \PY{o}{=} \PY{n}{xnew}\PY{p}{[}\PY{o}{\PYZhy{}}\PY{l+m+mi}{1}\PY{p}{]}
        \PY{k}{print} \PY{l+s+s1}{\PYZsq{}}\PY{l+s+s1}{\PYZhy{}\PYZhy{}\PYZhy{}\PYZhy{}\PYZhy{}\PYZhy{}\PYZhy{}\PYZhy{} Metodo de Newton \PYZhy{}\PYZhy{}\PYZhy{}\PYZhy{}\PYZhy{}\PYZhy{}\PYZhy{}\PYZhy{}}\PY{l+s+s1}{\PYZsq{}}
        \PY{k}{print} \PY{l+s+s1}{\PYZsq{}}\PY{l+s+s1}{Funcion = x\PYZca{}2 \PYZhy{} 4x}\PY{l+s+s1}{\PYZsq{}}
        \PY{c+c1}{\PYZsh{}try/except por si ocurre division por 0}
        \PY{k}{try}\PY{p}{:}
            \PY{k}{while} \PY{n+nb+bp}{True}\PY{p}{:}
                \PY{k}{print} \PY{l+s+s1}{\PYZsq{}}\PY{l+s+s1}{Mejor aproximacion = }\PY{l+s+s1}{\PYZsq{}}\PY{p}{,}\PY{n}{resp}
        \PY{c+c1}{\PYZsh{}metodo de newton\PYZhy{}raphson}
                \PY{n}{resp} \PY{o}{=} \PY{n}{resp} \PY{o}{\PYZhy{}} \PY{p}{(}\PY{n}{f1}\PY{p}{(}\PY{n}{resp}\PY{p}{)}\PY{o}{/}\PY{n}{df1}\PY{p}{(}\PY{n}{resp}\PY{p}{)}\PY{p}{)}
        \PY{c+c1}{\PYZsh{}agregar esta resp a xnew}
                \PY{n}{xnew}\PY{o}{.}\PY{n}{append}\PY{p}{(}\PY{n}{resp}\PY{p}{)}
        \PY{c+c1}{\PYZsh{}alcanzar el error exigido}
                \PY{k}{if} \PY{n+nb}{abs}\PY{p}{(}\PY{n}{xnew}\PY{p}{[}\PY{o}{\PYZhy{}}\PY{l+m+mi}{2}\PY{p}{]}\PY{o}{\PYZhy{}}\PY{n}{xnew}\PY{p}{[}\PY{o}{\PYZhy{}}\PY{l+m+mi}{1}\PY{p}{]}\PY{p}{)} \PY{o}{\PYZlt{}}\PY{o}{=} \PY{n}{erro}\PY{p}{:}
                    \PY{k}{break}
        \PY{k}{except} \PY{p}{(}\PY{n+ne}{ZeroDivisionError}\PY{p}{)}\PY{p}{:}
            \PY{k}{print} \PY{l+s+s1}{\PYZsq{}}\PY{l+s+s1}{Ha ocurrido una division por 0. No se puede continuar}\PY{l+s+s1}{\PYZsq{}}
        \PY{k}{print} \PY{l+s+s1}{\PYZsq{}}\PY{l+s+s1}{***La aproximacion final es}\PY{l+s+s1}{\PYZsq{}}\PY{p}{,}\PY{n}{resp} 
\end{Verbatim}


    El programa arroja:

\begin{verbatim}
-------- Metodo de Newton --------
Funcion = x^2 - 4x 
Mejor aproximacion =  3.0
Mejor aproximacion =  4.5
Mejor aproximacion =  4.05
Mejor aproximacion =  4.0006097561
***La aproximacion final es 4.00000009292 
\end{verbatim}

    \subsection{Solución apartado B)}\label{soluciuxf3n-apartado-b}

    \begin{Verbatim}[commandchars=\\\{\}]
{\color{incolor}In [{\color{incolor} }]:} \PY{c+c1}{\PYZsh{}para la funcion exponencial}
        \PY{k+kn}{from} \PY{n+nn}{math} \PY{k+kn}{import} \PY{n}{exp}
        \PY{c+c1}{\PYZsh{} definimos funciones}
        \PY{c+c1}{\PYZsh{} f2(x) \PYZhy{}\PYZhy{}\PYZhy{}\PYZgt{} e\PYZca{}x \PYZhy{} 4x }
        \PY{c+c1}{\PYZsh{} df2(x) \PYZhy{}\PYZhy{}\PYZhy{}\PYZgt{} e\PYZca{}x \PYZhy{} 4}
        \PY{k}{def} \PY{n+nf}{f2}\PY{p}{(}\PY{n}{x}\PY{p}{)}\PY{p}{:}
            \PY{k}{return} \PY{n}{exp}\PY{p}{(}\PY{n}{x}\PY{p}{)} \PY{o}{\PYZhy{}} \PY{l+m+mi}{4}\PY{o}{*}\PY{n}{x}
        \PY{k}{def} \PY{n+nf}{df2}\PY{p}{(}\PY{n}{x}\PY{p}{)}\PY{p}{:}
            \PY{k}{return} \PY{n}{exp}\PY{p}{(}\PY{n}{x}\PY{p}{)} \PY{o}{\PYZhy{}} \PY{l+m+mi}{4} 
        
        \PY{c+c1}{\PYZsh{}supuesto inicial}
        \PY{n}{xnew} \PY{o}{=} \PY{p}{[}\PY{l+m+mf}{3.0}\PY{p}{]}
        \PY{c+c1}{\PYZsh{}error tolerado al aproximar }
        \PY{n}{erro} \PY{o}{=} \PY{l+m+mf}{0.001} 
        \PY{c+c1}{\PYZsh{}raiz de la funcion}
        \PY{n}{resp} \PY{o}{=} \PY{n}{xnew}\PY{p}{[}\PY{o}{\PYZhy{}}\PY{l+m+mi}{1}\PY{p}{]}
        \PY{k}{print} \PY{l+s+s1}{\PYZsq{}}\PY{l+s+s1}{\PYZhy{}\PYZhy{}\PYZhy{}\PYZhy{}\PYZhy{}\PYZhy{}\PYZhy{}\PYZhy{} Metodo de Newton \PYZhy{}\PYZhy{}\PYZhy{}\PYZhy{}\PYZhy{}\PYZhy{}\PYZhy{}\PYZhy{}}\PY{l+s+s1}{\PYZsq{}}
        \PY{k}{print} \PY{l+s+s1}{\PYZsq{}}\PY{l+s+s1}{Funcion = e\PYZca{}x \PYZhy{} 4x}\PY{l+s+s1}{\PYZsq{}}
        \PY{c+c1}{\PYZsh{}try/except por si ocurre division por 0}
        \PY{k}{try}\PY{p}{:}
            \PY{k}{while} \PY{n+nb+bp}{True}\PY{p}{:}
                \PY{k}{print} \PY{l+s+s1}{\PYZsq{}}\PY{l+s+s1}{Mejor aproximacion = }\PY{l+s+s1}{\PYZsq{}}\PY{p}{,}\PY{n}{resp}
        \PY{c+c1}{\PYZsh{}metodo de newton\PYZhy{}raphson}
                \PY{n}{resp} \PY{o}{=} \PY{n}{resp} \PY{o}{\PYZhy{}} \PY{p}{(}\PY{n}{f2}\PY{p}{(}\PY{n}{resp}\PY{p}{)}\PY{o}{/}\PY{n}{df2}\PY{p}{(}\PY{n}{resp}\PY{p}{)}\PY{p}{)}
        \PY{c+c1}{\PYZsh{}agregar esta resp a xnew}
                \PY{n}{xnew}\PY{o}{.}\PY{n}{append}\PY{p}{(}\PY{n}{resp}\PY{p}{)}
        \PY{c+c1}{\PYZsh{}alvanzar el error exigido}
                \PY{k}{if} \PY{n+nb}{abs}\PY{p}{(}\PY{n}{xnew}\PY{p}{[}\PY{o}{\PYZhy{}}\PY{l+m+mi}{2}\PY{p}{]}\PY{o}{\PYZhy{}}\PY{n}{xnew}\PY{p}{[}\PY{o}{\PYZhy{}}\PY{l+m+mi}{1}\PY{p}{]}\PY{p}{)} \PY{o}{\PYZlt{}}\PY{o}{=} \PY{n}{erro}\PY{p}{:}
                    \PY{k}{break}
        \PY{k}{except} \PY{p}{(}\PY{n+ne}{ZeroDivisionError}\PY{p}{)}\PY{p}{:}
            \PY{k}{print} \PY{l+s+s1}{\PYZsq{}}\PY{l+s+s1}{Ha ocurrido una division por 0. No se puede continuar}\PY{l+s+s1}{\PYZsq{}}
        \PY{k}{print} \PY{l+s+s1}{\PYZsq{}}\PY{l+s+s1}{***La aproximacion final es}\PY{l+s+s1}{\PYZsq{}}\PY{p}{,}\PY{n}{resp} 
\end{Verbatim}


    El programa arroja:

\begin{verbatim}
-------- Metodo de Newton --------
Funcion = e^x - 4x
Mejor aproximacion =  3.0
Mejor aproximacion =  2.49734118533
Mejor aproximacion =  2.23221940087
Mejor aproximacion =  2.15860801401
Mejor aproximacion =  2.15331857522
***La aproximacion final es 2.15329236475
\end{verbatim}

    \subsection{Solución apartado C)}\label{soluciuxf3n-apartado-c}

    \begin{Verbatim}[commandchars=\\\{\}]
{\color{incolor}In [{\color{incolor} }]:} \PY{c+c1}{\PYZsh{}definimos funciones}
        
        \PY{c+c1}{\PYZsh{}f3(x)\PYZhy{}\PYZhy{}\PYZhy{}\PYZgt{}10x \PYZhy{} x\PYZca{}2}
        \PY{c+c1}{\PYZsh{}df3(x)\PYZhy{}\PYZhy{}\PYZhy{}\PYZgt{}10 \PYZhy{} 2x }
        \PY{k}{def} \PY{n+nf}{f3}\PY{p}{(}\PY{n}{x}\PY{p}{)}\PY{p}{:}
            \PY{k}{return} \PY{l+m+mi}{10}\PY{o}{*}\PY{n}{x} \PY{o}{\PYZhy{}} \PY{n}{x}\PY{o}{*}\PY{o}{*}\PY{l+m+mi}{2} 
        \PY{k}{def} \PY{n+nf}{df3}\PY{p}{(}\PY{n}{x}\PY{p}{)}\PY{p}{:}
            \PY{k}{return} \PY{l+m+mi}{10} \PY{o}{\PYZhy{}} \PY{l+m+mi}{2}\PY{o}{*}\PY{n}{x} 
        
        \PY{c+c1}{\PYZsh{}supuesto inicial}
        \PY{n}{xnew} \PY{o}{=} \PY{p}{[}\PY{l+m+mf}{7.0}\PY{p}{]}
        \PY{c+c1}{\PYZsh{}error tolerado al aproximar }
        \PY{n}{erro} \PY{o}{=} \PY{l+m+mf}{0.001} 
        \PY{c+c1}{\PYZsh{}raiz de la funcion}
        \PY{n}{resp} \PY{o}{=} \PY{n}{xnew}\PY{p}{[}\PY{o}{\PYZhy{}}\PY{l+m+mi}{1}\PY{p}{]}
        \PY{k}{print} \PY{l+s+s1}{\PYZsq{}}\PY{l+s+s1}{\PYZhy{}\PYZhy{}\PYZhy{}\PYZhy{}\PYZhy{}\PYZhy{}\PYZhy{}\PYZhy{} Metodo de Newton \PYZhy{}\PYZhy{}\PYZhy{}\PYZhy{}\PYZhy{}\PYZhy{}\PYZhy{}\PYZhy{}}\PY{l+s+s1}{\PYZsq{}}
        \PY{k}{print} \PY{l+s+s1}{\PYZsq{}}\PY{l+s+s1}{Funcion = 10 \PYZhy{} x\PYZca{}2}\PY{l+s+s1}{\PYZsq{}}
        \PY{c+c1}{\PYZsh{}try/except por si ocurre division por 0}
        \PY{k}{try}\PY{p}{:}
            \PY{k}{while} \PY{n+nb+bp}{True}\PY{p}{:}
                \PY{k}{print} \PY{l+s+s1}{\PYZsq{}}\PY{l+s+s1}{Mejor aproximacion = }\PY{l+s+s1}{\PYZsq{}}\PY{p}{,}\PY{n}{resp}
        \PY{c+c1}{\PYZsh{}metodo de newton\PYZhy{}raphson}
                \PY{n}{resp} \PY{o}{=} \PY{n}{resp} \PY{o}{\PYZhy{}} \PY{p}{(}\PY{n}{f3}\PY{p}{(}\PY{n}{resp}\PY{p}{)}\PY{o}{/}\PY{n}{df3}\PY{p}{(}\PY{n}{resp}\PY{p}{)}\PY{p}{)}
        \PY{c+c1}{\PYZsh{}agregar esta resp a xnew}
                \PY{n}{xnew}\PY{o}{.}\PY{n}{append}\PY{p}{(}\PY{n}{resp}\PY{p}{)}
        \PY{c+c1}{\PYZsh{}alcanzar el error exigido}
                \PY{k}{if} \PY{n+nb}{abs}\PY{p}{(}\PY{n}{xnew}\PY{p}{[}\PY{o}{\PYZhy{}}\PY{l+m+mi}{2}\PY{p}{]}\PY{o}{\PYZhy{}}\PY{n}{xnew}\PY{p}{[}\PY{o}{\PYZhy{}}\PY{l+m+mi}{1}\PY{p}{]}\PY{p}{)} \PY{o}{\PYZlt{}}\PY{o}{=} \PY{n}{erro}\PY{p}{:}
                    \PY{k}{break}
        \PY{k}{except} \PY{p}{(}\PY{n+ne}{ZeroDivisionError}\PY{p}{)}\PY{p}{:}
            \PY{k}{print} \PY{l+s+s1}{\PYZsq{}}\PY{l+s+s1}{Ha ocurrido una division por 0. No se puede continuar}\PY{l+s+s1}{\PYZsq{}}
        \PY{k}{print} \PY{l+s+s1}{\PYZsq{}}\PY{l+s+s1}{***La aproximacion final es}\PY{l+s+s1}{\PYZsq{}}\PY{p}{,}\PY{n}{resp} 
\end{Verbatim}


    El programa arroja:

\begin{verbatim}
-------- Metodo de Newton --------
Funcion = 10 - x^2
Mejor aproximacion =  7.0
Mejor aproximacion =  12.25
Mejor aproximacion =  10.349137931
Mejor aproximacion =  10.0113941065
Mejor aproximacion =  10.000012953
***La aproximacion final es 10.0
\end{verbatim}


    % Add a bibliography block to the postdoc
    
    
    
    \end{document}
